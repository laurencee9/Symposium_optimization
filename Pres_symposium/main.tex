\documentclass{beamer}
\usetheme{sthlm}
\usepackage{subfigure}
\usepackage{braket}
\usepackage{xcolor}
\usepackage{bm}
\usepackage[utf8]{inputenc}
\usepackage[T1]{fontenc}
\usepackage[francais]{babel}
\usepackage{listings}
\usepackage{transparent}

\definecolor{myblue}{HTML}{E3F3DA}

\definecolor{darkorange}{HTML}{A62500}
\definecolor{myred}{HTML}{ECCDC1}
% \highlight[<colour>]{<stuff>}
\newcommand{\highlight}[2][yellow]{\mathchoice%
  {\colorbox{#1}{$\displaystyle#2$}}%
  {\colorbox{#1}{$\textstyle#2$}}%
  {\colorbox{#1}{$\scriptstyle#2$}}%
  {\colorbox{#1}{$\scriptscriptstyle#2$}}}%


\newcommand{\ins}[0]{\mathrm{in}}
\newcommand{\out}[0]{\mathrm{out}}
\def\Put(#1,#2)#3{\leavevmode\makebox(0,0){\put(#1,#2){#3}}}
\newcommand{\info}[1]{\begin{itemize}\item[ ]\tiny{#1}\end{itemize}}
\newcommand{\reference}[1]{\tiny{#1}}

\newcolumntype{L}[1]{>{\raggedright\let\newline\\\arraybackslash\hspace{0pt}}m{#1}}
\newcolumntype{C}[1]{>{\centering\let\newline\\\arraybackslash\hspace{0pt}}m{#1}}
\newcolumntype{R}[1]{>{\raggedleft\let\newline\\\arraybackslash\hspace{0pt}}m{#1}}
 

\newcommand{\BStructure}{\,^{(S)}B}
\newcommand{\BNode}{\,^{(N)}B}
\newcommand{\GStructure}{\,^{(S)}G}
\newcommand{\GNode}{\,^{(N)}G}
\newcommand{\XB}{\,^{(X)}B}
\newcommand{\XG}{\,^{(X)}G}
\newcommand{\SB}{\,^{(S)}B}
\newcommand{\SG}{\,^{(S)}G}
\newcommand{\NB}{\,^{(N)}B}
\newcommand{\NGown}{\,^{(N)}G}
\newcommand{\NSG}{\,^{(N,S)}G}
\newcommand{\NSB}{\,^{(N,S)}B}



% Décommenter pour avoir une numérotation subtile des pages.
% 
\addtobeamertemplate{navigation symbols}{}{%
    \usebeamerfont{footline}%
    \usebeamercolor[fg]{footline}%
    \hspace{1em}%
    \vspace{0.1cm}
    \large{\insertframenumber}
}

% ---------------------------------------------------------------
% Page titre
% ---------------------------------------------------------------
\title{L'optimisation}
\subtitle[Sous-titre court]{Une revue}
\author{Edward Laurence \& Guillaume St-Onge}
\institute{Département de physique, de génie physique, et d'optique\\ Université Laval, Québec, Canada}
\date{11 avril 2016}
\begin{document}
\begin{frame}
  \titlepage
\end{frame}


\begin{frame}{Optimisation}
  
\end{frame}

\begin{frame}{Type d'algorithmes}
\textbf{Heuristique}\\
  Spécialisé à un problème et ne garantit pas la solution obtenue.\\
\vspace{1cm}

\textbf{Métaheuristique}\\
  Algorithme général qu'on doit adapter au problème considéré.

\end{frame}

\begin{frame}{Recherche tabou}
 \textbf{Recherche Tabou}\\
  \textit{Type : }Métaheuristique\\
  \textit{Stochastique : } Non\\
  \textit{Caractéristique : } Recherche local
  \vspace{0.5cm}
\hrule
\vspace{0.2cm}
\textbf{Principes}\\
1. On recherche le mouvement qui minimise notre fonction.\\
2. On ne revient pas sur nos pas (d'où \textit{tabou}).  
\end{frame}



\begin{frame}{Exemple - Recherche tabou}
  \textit{On cherche à descendre de la montagne.}
  \begin{figure}[tb]
    \centering
    \includegraphics<1>[width=0.7\textwidth]{figures/tabou1.pdf}
    \includegraphics<2>[width=0.7\textwidth]{figures/tabou2.pdf}
    \includegraphics<3>[width=0.7\textwidth]{figures/tabou3.pdf}
  \end{figure}
\end{frame}

\begin{frame}{Algorithme des lucioles}
   \textbf{Recherche par lucioles}\\
  \textit{Type : }Métaheuristique\\
  \textit{Stochastique : } Oui\\
  \textit{Caractéristique : } Recherche globale
  \vspace{0.5cm}
  \hrule
\vspace{0.2cm}
\textbf{Principes}\\
2. Chaque luciole a une luminosité $I$ et une position.\\
3. Les lucioles sont attirées par les lucioles plus lumineuses.\\
4. L'attirance décroît lorsque la distance augmente.
\end{frame}

\begin{frame}{Algorithme des lucioles}
  $N$ lucioles à des positions $\bm{x}_i$\\
  On optimise la fonction $f(\bm{x})$
  \vspace{0.5cm}
  \hrule 
  \begin{align*}
    x_i \rightarrow x_{i} +  \highlight[myblue]{\beta_0\text{e}^{-\gamma r_{ij}^2}(\bm{x}_j-\bm{x}_i)}+ \highlight[myred]{\alpha\epsilon_i}
  \end{align*}
  $\beta_0=0$ : Marche aléatoire\\
  ($\gamma=0$ : Optimisation par essaims particulaires)
\end{frame}

% \begin{frame}{Pseudocode - Algorithme des lucioles}
%  1) Placer $N$ lucioles à différents endroits $\bm{x}_i$\\
%  2) Mesurer l'intensité des lucioles $I_i \propto f(\bm{x}_i)$\\
%  3) $\gamma$ : Coefficient d'absorption\\
%  4) $\beta_0$ : Attractivité

%  \textbf{while} (temps<temps\_max):\\
% \hspace{0.5cm} \textbf{for} $i=1..N$:\\
% \hspace{1.0cm} \textbf{for} $j=1..N$:\\
% \hspace{1.5cm} \textbf{if} $f(\bm{x}_j)>f(\bm{x}_i)$\\
% \hspace{2.0cm} Déplacer $i$ vers $j$\\
% \hspace{1.5cm} \textbf{end if}\\
% \hspace{1.cm} \textbf{end for}\\
% \hspace{0.5cm} \textbf{end for}
% Mise à jour des intensités.
% \end{frame}


\begin{frame}{Exemple - Algorithme des lucioles}
  \textit{On cherche à descendre de la montagne.}
  \begin{figure}[tb]
    \centering
    \includegraphics<1>[width=0.7\textwidth]{figures/firefly1.pdf}
    \includegraphics<2>[width=0.7\textwidth]{figures/firefly2.pdf}
  \end{figure}
\end{frame}


\begin{frame}{Comparaison des algorithmes}
  
\end{frame}

\begin{frame}{Problème du vendeur}
  
\end{frame}


\end{document}
